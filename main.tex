\documentclass[12pt]{article}
\usepackage[utf8]{inputenc}
\usepackage{amsmath}
\usepackage{amsfonts}
%\usepackage[dvipdfmx]{hyperref,graphicx}
\usepackage{graphicx}
\usepackage{apacite}

\usepackage[colorlinks,citecolor=red,urlcolor=blue,bookmarks=false,hypertexnames=true]{hyperref,natbib} 
\usepackage[top=15truemm]{geometry}
%\usepackage[]{natbib} 
\bibpunct[:]{(}{)}{,}{a}{}{,}
\setlength{\textwidth}{\paperwidth}     % ひとまず紙面を本文領域に
\setlength{\oddsidemargin}{-5.4truemm}  % 左の余白を20mm(=1inch-5.4mm)に
\setlength{\evensidemargin}{-5.4truemm} % 
\addtolength{\textwidth}{-40truemm}     % 右の余白も20mmに
\renewcommand{\baselinestretch}{0.3}

\usepackage{setspace}
\setstretch{1.2}

\title{An MPEC Estimator for Sequential Search model}
\author{Suguru Otani}
\date{December 2023}

\begin{document}

\maketitle

\section{Introduction}

Consumer search process to gather information about possible options is prevalent in the actual choice behavior.
The search process has increasingly become observable to researchers. 
For example, browsing data allows the researcher to observe which options an agent inspects before making a choice. 
The increased availability of consumer search data has led researchers to estimate structural models of consumer search behavior, summarized in \cite{ursu2023sequential}. 
The current benchmark model nests the inner loop step to solve implicit functions regarding reserved prices of searching decisions as a fixed point problem, which is known to be computationally burdensome.

To overcome the problem, we propose a new estimator based on the mathematical program
with equilibrium constraints (MPEC) approach advocated by \cite{su2012constrained}. 
MPEC is a constrained optimization problem whose constraint structure contains the equilibrium constraints.
The main advantage of MPEC is that MPEC does not need to solve the fixed point problem iteratively, which is well known in the demand estimation \citep{dube2012improving}, dynamic programming \citep{su2012constrained,egesdal2015estimating}, and misclassification model \citep{lu2014mpec}.
Based on Monte Carlo simulations, we show that MPEC estimator works better for the sequential search model relative to the nested fixed point approaches and the advantage carries over. 
We also show that increasing the sample size improves the accuracy of estimation. 
We conclude that XXX.

\section{Weitzman's sequential search model}

\subsection{The Weitzman (1979) Framework}
I construct the sequential
search model based on \cite{weitzman1979optimal}.
See the detail in the unified survey of \cite{ursu2023sequential} and \cite{honka2019empirical}. 
A decision maker $i$ faces a set of $\mathcal{J}=\{1,\cdots,J\}$ boxes and box $j$ gives a potential reward $u_{ij}\in \mathbb{R}$ independently drawn from a known
distribution $F_{ij}(u)$.
Opening box $j$ takes cost $c_{ij}\in \mathbb{R}^{++}$. 
An outside option is denoted as $j = 0$ with a known reward $u_{i0}$ is available at no cost.
The decision maker
opens boxes via sequential search steps and her goal is to maximize her expected reward net of total costs.

Suppose that the decision maker has opened a set \(S_{i}\) of boxes,
which revealed a maximum reward value of \(u_{i}^{*}=\max _{j \in S_{i} \cup 0} u_{i j}\), and \(\bar{S}_{i}\) unopened boxes can still be
opened. 
Her dynamic programming problem choosing to stop opening boxes, in which case she
gets payoff \(u_{i}^{*}\), or to continue opening boxes is described
by the following Bellman equation:
\begin{align}
    V\left(\bar{S}_{i}, u_{i}^{*}\right)=\max \left\{u_{i}^{*}, \max _{j \in \bar{S}_{i}}\left\{-c_{i j}+W_{j}\left(\bar{S}_{i}, u_{i}^{*}\right)\right\}\right\}
\end{align}
where \(W_{j}\left(\bar{S}_{i}, u_{i}^{*}\right)\) is the expected value of continuing to open boxes and is
defined as
\begin{align}
    W_{j}\left(\bar{S}_{i}, u_{i}^{*}\right)=V\left(\bar{S}_{i} \backslash j, u_{i}^{*}\right) \int_{-\infty}^{u_{i}^{*}} d F_{i j}(u)+\int_{u_{i}^{*}}^{\infty} V\left(\bar{S}_{i} \backslash j, u\right) d F_{i j}(u).
\end{align}
Also, the reservation utility of a product \(z_{i j}\) is the utility level defined as 
\begin{align}
    \int_{z_{i j}}^{\infty}\left(u_{i j}-z_{i j}\right) d F_{i j}\left(u_{i j}\right)=c_{i j}.
\end{align}

A set of optimal decision rules, developed by
\cite{weitzman1979optimal}, is used to characterize consumers' optimal search and choice strategies. 
The rules are as follows:
\begin{enumerate}
    \item Consumers know the true distribution(s) \(F_{i j}(u)\).
    \item Search fully reveals the utility associated with product \(j\).
    \item For each consumer \(i, u_{i j}\) is independently (across \(j\) ) drawn from \(F_{i j}(u)\).
\end{enumerate}
Then, the optimal search and choice decision
rules are expressed as follows:
\begin{enumerate}
    \item Selection Rule: The consumer searches in decreasing order of reservation utilities.
    \item Stopping Rule: Search terminates when the maximum observed utility exceeds the reservation utility of any unsearched product.
    \item Choice Rule: Once the consumer stops searching, she chooses the product with the highest observed utility among all searched options.
\end{enumerate}

\subsection{Parametrizations for Empirical Work}

Empirical economists often assume consumer $i$'s utility from product $j$ has two additively separable components:
\begin{align}
    u_{i j} & =\delta_{i j}+\varepsilon_{i j}  =\left(\xi_{i j}+\mu_{i j}\right)+\varepsilon_{i j},\\
    \quad \quad \varepsilon_{i j}&\sim_{i.i.d} N(0,\sigma_{\mu}),\quad \mu_{i j} \sim_{i.i.d} N(0,\sigma_{\varepsilon})\nonumber
\end{align}
where \(\delta_{i j}\) is utility which is known by the consumer prior to search ("pre-search
utility" in the following) and \(\varepsilon_{i j}\) is utility that is only known by the consumer
after search ("post-search taste shock" in the following). 
We assume that the pre-search utility
\(\delta_{i j}\) consists of a component \(\xi_{i j}\) that can be observed by the researcher and a pre-search taste shock \(\mu_{i j}\) that
cannot be observed by the researcher. 
According to \cite{ursu2023sequential}, we need to further normalize their variance by setting $\sigma_{\mu}= \sigma_{\varepsilon} =1$. 

Under the assumption of normally distributed post-search taste shocks, we can derive the following
expression for the reservation utility:
\begin{align}
    z_{i j}=\delta_{i j}+m\left(c_{i j}\right)=\xi_{i j}+\mu_{i j}+m\left(c_{i j}\right) 
\end{align}
where \(m\left(c_{i j} \right)\) is the implicit function that solves the following equation (see \cite{kim2010online}):
\begin{align}
    c_{i j}=\phi(m)+m \times[\Phi(m)-1] \label{eq:equilibrium_constraint}
\end{align}
with \(\phi\) and \(\Phi\) denoting the standard normal pdf and cdf, respectively. \cite{weitzman1979optimal} shows the existence and uniqueness of the solution of \eqref{eq:equilibrium_constraint}.

There are four methods to solve \eqref{eq:equilibrium_constraint}. 
A first approach, proposed by \cite{kim2010online}, is pre-computing the mapping between $m$ and $c$ and saving it in a look-up table.
A second approach proposed by \cite{jiang2021consumer} utilizes Newton's method to compute reservation utilities by successively better approximations to the root
of a function 
\begin{align*}
    q(m)=(1-\Phi(m))\left(\frac{\phi(m)}{1-\Phi(m)}-m\right)-c=0.
\end{align*}
A third approach proposed by \cite{elberg2019dynamic} is to use a contraction mapping of
\begin{align*}
    \Gamma(m)=-c+\phi(m)+m \times \Phi(m).
\end{align*}
A fourth approach proposed by \cite{morozov2023measuring} is to directly estimate $m(c_{ij})$.

\cite{ursu2023sequential} mention that (1) the first method introduces error from using linear interpolations for search cost
values that are not equal to grid-point values, (2) the second and third methods avoid any error from using linear interpolation but involve iterations of finding $m$ and defining a convergence threshold which can result in numerical errors when the threshold is too loose, (3) the fourth approach involves estimation error on $m(c_{ij})$. 
In practice, both methods appear to converge quickly and allow researchers to set tight convergence thresholds thus avoiding numerical problems \citep{ursu2023sequential}. 
Similar problems on demand estimation are discussed and resolved by MPEC introduced next \citep{dube2012improving}. 



\section{An MPEC estimator for the sequential search model}
As the fifth method, I propose a novel and simple estimator for the sequential search model by utilizing
the Mathematical Programming with Equilibrium Constraints (MPEC) procedure advocated
by \cite{su2012constrained}. 
Our setting is similar to \cite{su2012constrained} in which the likelihood function
function with equilibrium constraints is constructed. 
MPEC estimator avoids the iterative computations to find the fixed point by evaluating the equilibrium equations as constraints.

Define $\theta$ as a set of parameters.
Given the optimal decision rules and reservation utility formula, an MPEC estimator solves the following constrained problem:
\begin{align}
    \max_{\theta}& \sum_{i\in \mathcal{N}} \log L_{i}(\theta,(z_{ij})_{j\in \mathcal{J}},(u_{ij})_{j\in \mathcal{J}})\nonumber\\
    \text{s.t.}\quad u_{i j} & =\xi_{i j}+\mu_{i j}+\varepsilon_{i j}\label{eq:mpec_formula}\\
    z_{i j}&=\xi_{i j}+\mu_{i j}+m\left(c_{i j}\right) \nonumber\\
    c_{i j}&=\phi(m)+m \times[\Phi(m)-1] \nonumber
\end{align}
where individual likelihood $L_{i}(\theta,(z_{ij})_{j\in \mathcal{J}},(u_{ij})_{j\in \mathcal{J}})$ is derived in Appendix.

For comparison, I Follow \cite{ursu2023sequential} on the setting of the parameters of $\xi_{ij}$ and $c_{ij}$ in \eqref{eq:mpec_formula}. I compare their above first approach with MPEC. Table XX shows that MPEC estimator achieve better performance than the first approach without any ad hoc imposed assumption in a look-up table. However, computational time is longer than the first approach.

\section{Conclusion}

\appendix
\section{Derivation of individual likelihood $L_{i}(\theta,(z_{ij}(\theta))_{j\in \mathcal{J}})$}

Following \cite{ursu2023sequential}, I construct individual likelihood $L_{i}(\theta,(z_{ij}(\theta))_{j\in \mathcal{J}})$ as follows.


\bibliographystyle{aer}
\bibliography{sequential_search}

\end{document}
